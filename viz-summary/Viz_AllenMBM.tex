\documentclass[a4paper]{article}
 
\usepackage{amsmath}
\usepackage{graphicx}
\usepackage{caption}
\usepackage{subfigure}
\usepackage{epstopdf}
\usepackage[ansinew]{inputenc}
\usepackage{listings}
\usepackage{xcolor}
%\setlength{\oddsidemargin}{0cm}
%\setlength{\evensidemargin}{0cm}
%\setlength{\topmargin}{0cm}

\usepackage[]{algorithm2e}

\usepackage{a4wide}

\title{Visualize Allen mouse brain model}
\author{BBP}
%\date{}

\begin{document}
   \maketitle

\section{Introduction}
The Neurorobotics group combined data from the Allen institute for Brain science to produce a point-neuron model of the mouse brain.
Running this model in NEST is a first try to simulate a full characterized mouse brain model based on the Allen Brain Atlas.
To make it run NEST has to be adapted to different strategies of loading data.
Further new visualization strategies have to be investigated to analyze the behavior of this model.
Besides the large amount of data which has to be processed, the data has to be streamed or stored to disk for processing,
because the simulation by itself requires full computation and memory capacity of the cluster nodes.
\newline
The 	innovation of the full characterized mouse brain model is the integration of the point-neuron model with real coordinate information based on in vitro measurements. Additionally synapse information are given for all neurons.
Thus an exact geometrical representation of the point-neuron model in a 3D manner is possible.
The button-up approach makes a mapping of single simulated neuron quantities on the geometrical representation possible: 
E.g. mapping spiking activity on the 3D representation allows to visualize dynamic spiking propagation on different scales.
Thus you could realize a zoom from the whole brain model down to single neurons.
As the point neuron model is given a 3D representation could be rendered ones.




%\bibliographystyle{plain}

%\bibliography{./Viz_AllenMBM}

\end{document}
