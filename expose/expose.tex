\documentclass[a4paper]{article}
 
\usepackage{amsmath}
\usepackage{graphicx}
\usepackage{caption}
\usepackage{subfigure}
\usepackage{epstopdf}
\usepackage[ansinew]{inputenc}
\usepackage{listings}
\usepackage{xcolor}
%\setlength{\oddsidemargin}{0cm}
%\setlength{\evensidemargin}{0cm}
%\setlength{\topmargin}{0cm}

\usepackage[]{algorithm2e}

\usepackage{a4wide}

\title{Mouse Brain Model Close to Real Time Simulation - Expose}
\author{Till Schumann}
%\date{}

\begin{document}
   \maketitle


\section{Introduction}
In 2014, the BBP Neurorobotics team successfully produced a point-neuron model
of a mouse brain from data provided by the Allen Institute for Brain Science.
Using NEST software, the BBP Neurorobotics team was then able to simulate a
scaled-down model of the mouse brain on a laptop computer, paving the way towards
making a mouse brain interact with an environment. A full-scale implementation of
the mouse brain model is possible, but requires the use of super-computers.
The current functionality of NEST supports  generating large neuronal networks
on a super computer based an random distributions. For the mouse brain model
exact point neurons and connections information are available. The information should
be used to load the mouse brain model to NEST. Therefore the network creation
functionality of NEST has to be extended to support efficient parallel IO.
Currently the point-neuron model of the mouse brain is stored in HDF5 files.
The underlaying data structure was created to load the data to laptop computers.
Considering super-computers the data structure of the files should be
validated in respect to efficiency and potentially adapted to new requirements.

%The BBP HPC team has extensive knowledge on developing and optimizing applications for large core counts.

\section{NEST implementation}

The neuronal spiking network simulator NEST is developed in \emph{C++} and delivers
an user interface based on an own description language \emph{SLI} and  and a Python interface.
The new use case shell be integrated into the standard work flow of NEST.
Besides the functionality in \emph{C++} the interfaces have to be extended.
The difficulties of the network generation is based on a difference in 
the NEST internal data structure and the data delivered by the Allien Institute.
Connection information contains target and source neurons besides biochemical
information of the synapses. Because of the in vitro injection methods the
connection information maps the synapse from the source to the target neurons.
For multi process simulations NEST distributes all neurons based on a modulo function 
to the processes. Because of memory optimizations the synapses are only stored on the
post synaptic process. Which means that the connection information is stored
on the process, where the target neuron is located. Therefore a transformation of the given data is
necessary. Preprocessing of the input data should be avoided as far as possible to capture
future use cases.
The resulting implementation shell load the connection information efficiently in parallel,
distribute the synapse information to the post synaptic node and store it in
the NEST data structure.
Further requirements of the implementation are an efficient use of the available resources as
memory and computation power. As the implementation runs during the build up process of the
NEST network, the implementation has to handle variations of the available resources. 

\newpage
\section{Road map}
\begin{enumerate}
    \item Analyze existing code of scaled-down model of the mouse brain from the Neurorobotics team
    \item Define requirements of the new implementation in terms of memory consumption and computation power
    \item Define future requirements to mouse, rat and cat brain models using NEST
    \item Validate and eventually adapt HDF5 file format
    \item Develop algorithm to distribute connection information
    \item Implement Algorithm and test on the Blue Gene Q
    \item Integrate Algorithm in the NEST work flow
    \item Optimization of code execution to reduce time to solution of mouse brain model on the Blue Gene Q
\end{enumerate}

  

%In 2014, they extracted the core functionalities of NEURON simulation software into a new software referred as coreNeuron, reducing the number of lines of the code from 300k to 15k. In addition, they were able %%to greatly improve the performance of the new implementation by using OpenMP, full vectorization of the kernels, reaching almost maximum performance of the full JUQUEEN system (458752 cores, 29360128 threads).


\section{Summary}
This master thesis aims at producing a full-scale point-neuron model of the mouse
brain, based on the preparatory work of the Neurorobotics team. To this end, the
current tool-chain will be ported to the BlueGene IV system. To reduce time to
solution, optimization opportunities will be investigated and implemented whenever
possible.
Finally, we expect this master thesis to support the definition of requirements for the
Human Brain Project HPC platform, both in terms of networking and future system design.

%\newpage
%\bibliographystyle{plain}

%\bibliography{./expose}

\end{document}
